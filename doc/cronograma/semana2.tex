\documentclass[12pt,a4paper]{article}
\usepackage[margin=2cm]{geometry}
\usepackage[spanish]{babel}
\usepackage{graphicx, multicol, latexsym, amsmath, amssymb}
\usepackage{array}
\usepackage{tabularx}
\usepackage{setspace}
\usepackage{titlesec}
\usepackage{subcaption}
\usepackage{float}

% Ajuste de espacio entre secciones
\titlespacing*{\section}{0pt}{0.5em}{0.5em}

\renewcommand{\arraystretch}{1.3}

\begin{document}

% ---------------------------
% Encabezado con datos
% ---------------------------
\noindent
\begin{tabularx}{\textwidth}{|X|X|}
\hline
\textbf{Carné:} 2020045294 & \textbf{Nombre:} Justin Jaffeth Corea Masis \\
\hline
\textbf{Correo:} coreajustin288@estudiantec.cr & \textbf{Semana:} 2 del 13/08/2025 al 20/08/2025 \\
\hline
\end{tabularx}

\vspace{0.5cm}

% ---------------------------
% Tabla de actividades
% ---------------------------
\section*{Actividades realizadas}

% \begin{tabularx}{\textwidth}{|>{\raggedright\arraybackslash}p{8cm}|c|c|}
\begin{tabularx}{\textwidth}{|>{\raggedright\arraybackslash}p{12cm}|c|c|}
\hline
\textbf{Actividad} & \textbf{Fecha} & \textbf{Horas} \\
\hline
Inicié recorte de caracteres para dejarlos con fondo transparente & 14/08/2025 & 4 \\
\hline
Continué con recorte de caracteres para dejarlos con fondo transparente& 15/08/2025 & 2 \\
\hline
Inicié coloreado de caracteres a color azul RGB=(0,0,255) & 15/08/2025 & 4 \\
\hline
Continué con coloreado de caracteres a color azul RGB=(0,0,255) & 16/08/2025 & 2 \\
\hline
Agregué al código para generar placas el coloreado de caracteres según tipo de placa & 16/08/2025 & 2 \\
\hline
Inicie la redacción del capítulo 1 introducción y propuesta de la tesis & 16/08/2025 & 2 \\
\hline
Inicie la redacción del capítulo 1 introducción y propuesta de la tesis & 17/08/2025 & 6 \\
\hline
Corrección de bugs en código de generación de placas e hice preparativos para ejecutar los entrenamientos
	de YOLO. & 18/08/2025 & 2 \\
\hline
Agregué al código la posibilidad de agregar tornillos y el recorte de tornillos & 19/08/2025 & 3 \\
\hline 
Agregué al código la posibilidad de aplicar color no uniforme y ruidos a los caracteres y plantilla de placa & 19/08/2025 & 3 \\
\hline 
\end{tabularx}

\vspace{1cm}

% ---------------------------
% Firma
% ---------------------------
\noindent
\textbf{Firma del Profesor Asesor:}

\vspace{2cm}

\noindent\rule{8cm}{0.4pt}  
Nombre y firma

\newpage

\section{Ejemplos de caracteres}

\begin{figure}[H]
  % --- Fila 1 ---
  \begin{subfigure}[h]{0.32\textwidth}
    \includegraphics[width=0.2\linewidth]{../../src/font/results/structure/chars/blue/H/H_1.png}
  \end{subfigure}
  \begin{subfigure}[h]{0.32\textwidth}
    \includegraphics[width=0.2\linewidth]{../../src/font/results/structure/chars/blue/A/A_1.png}
  \end{subfigure}
  \begin{subfigure}[h]{0.32\textwidth}
    \includegraphics[width=0.2\linewidth]{../../src/font/results/structure/chars/blue/X/X_1.png}
  \end{subfigure}

  % --- Fila 2 ---
  \begin{subfigure}[h]{0.32\textwidth}
    \includegraphics[width=0.2\linewidth]{../../src/font/results/structure/chars/blue/B/B_1.png}
  \end{subfigure}
  \begin{subfigure}[h]{0.3\textwidth}
    \includegraphics[width=0.2\linewidth]{../../src/font/results/structure/chars/blue/8/8_1.png}
  \end{subfigure}
  \begin{subfigure}[h]{0.32\textwidth}
    \includegraphics[width=0.2\linewidth]{../../src/font/results/structure/chars/blue/2/2_1.png}
  \end{subfigure}
\end{figure}

\section{Ejemplos de placas sin aumento}

\begin{figure}[H]
	\begin{subfigure}[h]{0.3\textwidth}
		\includegraphics[width=\linewidth]{../../src/sources/images/mejores/raw/synthetic_plate_343.png}
	\end{subfigure}

	\begin{subfigure}[h]{0.3\textwidth}
		\includegraphics[width=\linewidth]{../../src/sources/images/mejores/raw/synthetic_plate_68.png}
	\end{subfigure}

	\begin{subfigure}[h]{0.3\textwidth}
		\includegraphics[width=\linewidth]{../../src/sources/images/mejores/raw/synthetic_plate_433.png}
	\end{subfigure}
\end{figure}

\section{Ejemplos de placas con muy ligero aumento}

\begin{figure}[H]
	\begin{subfigure}[h]{0.3\textwidth}
		\includegraphics[width=\linewidth]{../../src/sources/images/mejores/augmented/synthetic_plate_91.png}
	\end{subfigure}

	\begin{subfigure}[h]{0.3\textwidth}
		\includegraphics[width=\linewidth]{../../src/sources/images/mejores/augmented/synthetic_plate_221.png}
	\end{subfigure}

	\begin{subfigure}[h]{0.3\textwidth}
		\includegraphics[width=\linewidth]{../../src/sources/images/mejores/augmented/synthetic_plate_219.png}
	\end{subfigure}
\end{figure}

\end{document}
