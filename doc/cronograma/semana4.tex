\documentclass[12pt,a4paper]{article}
\usepackage[margin=2cm]{geometry}
\usepackage[spanish]{babel}
\usepackage{graphicx, multicol, latexsym, amsmath, amssymb}
\usepackage{array}
\usepackage{tabularx}
\usepackage{setspace}
\usepackage{titlesec}
\usepackage{subcaption}
\usepackage{float}
\usepackage{url}
\usepackage{hyperref}

% Ajuste de espacio entre secciones
\titlespacing*{\section}{0pt}{0.5em}{0.5em}

\renewcommand{\arraystretch}{1.3}

\begin{document}

% ---------------------------
% Encabezado con datos
% ---------------------------
\noindent
\begin{tabularx}{\textwidth}{|X|X|}
\hline
\textbf{Carné:} 2020045294 & \textbf{Nombre:} Justin Jaffeth Corea Masís \\
\hline
\textbf{Correo:} coreajustin288@estudiantec.cr & \textbf{Semana:} 4 del 27/08/2025 al 3/09/2025 \\
\hline
\end{tabularx}

\vspace{0.5cm}

% ---------------------------
% Tabla de actividades
% ---------------------------
\section*{Actividades realizadas}

% \begin{tabularx}{\textwidth}{|>{\raggedright\arraybackslash}p{8cm}|c|c|}
\begin{tabularx}{\textwidth}{|>{\raggedright\arraybackslash}p{12cm}|c|c|}
\hline
\textbf{Actividad} & \textbf{Fecha} & \textbf{Horas} \\
\hline
	Implementación de pruebas de YOLO con placas de vehículos de otros países. & 29/08/2025 & 4 \\
\hline
	Reentrenar YOLO con un conjunto de datos más grande y variado y hacer comparación de resultados. & 30/08/2025 & 4 \\
\hline
	Implementar el sistema de reconocimiento de caracteres (OCR)
	para extraer la información de las placas detectadas. & 31/08/2025 & 8 \\
\hline
	Entrenar versiones alternativas de OCR y ejecutar pruebas comparativas con W\&B. & 01/09/2025 & 8 \\
\hline
	Integrar módulos en un prototipo de App. & 02/09/2025 & 4 \\
\hline
	Documentar los resultados obtenidos hasta el momento. & 02/09/2025 & 4 \\
\hline
\end{tabularx}

\vspace{1cm}

% ---------------------------
% Firma
% ---------------------------
\noindent
\textbf{Firma del Profesor Asesor:}

\vspace{2cm}

\noindent\rule{8cm}{0.4pt}  
Nombre y firma

\newpage

\section{Detección de placas}
\subsection{Comparación de modelos}

\begin{figure}[H]
	\centering
	\includegraphics[width=0.7\linewidth]{../../src/runs/eval/new_model2/confusion_matrix.png}
	\caption{Desempeño del nuevo modelo ante imágenes de varios países}
	\label{fig:yolo_new}
\end{figure}

\begin{figure}[H]
	\centering
	\includegraphics[width=0.7\linewidth]{../../src/runs/eval/old_model2/confusion_matrix.png}
	\caption{Desempeño del modelo original ante imágenes de varios países}
	\label{fig:yolo_old}
\end{figure}

\subsection{Ejemplos de detección comparativa entre etiquetas y predicciones} 

\begin{figure}[H]
	\centering
	\begin{subfigure}[b]{0.45\textwidth}
		\centering
		\includegraphics[width=\linewidth]{/home/akai/Pictures/001cdd25e148cd36.jpg}
		\caption{Fallo total. Vió una placa donde no había. Falló dos veces placas}
		\label{fig:det_fail1}
	\end{subfigure}
	\hfill
	\begin{subfigure}[b]{0.45\textwidth}
		\centering
		\includegraphics[width=\linewidth]{/home/akai/Pictures/0021af0b921af690.jpg}
		\caption{Fallo total. No detectó la placa.}
		\label{fig:det_fail2}
	\end{subfigure}

	\begin{subfigure}[b]{0.45\textwidth}
		\centering
		\includegraphics[width=\linewidth]{/home/akai/Pictures/IMG_20250625_102611_753.jpg}
		\caption{Fallo parcial. Detectó una placa, pero no la otra.}
		\label{fig:det_fail3}
	\end{subfigure}
	\hfill
	\begin{subfigure}[b]{0.45\textwidth}
		\centering
		\includegraphics[width=\linewidth]{/home/akai/Pictures/IMG_20250625_102730_391.jpg}
		\caption{Caso éxito. Detectó tres placas. Una no estaba etiquetada.}
		\label{fig:det_success}
	\end{subfigure}

	\caption{Ejemplos de detección de placas}
	\label{fig:eje_det}
\end{figure}

\section{Reconocimiento de caracteres}

\subsection{Curvas de entrenamiento}

\begin{figure}[H]
	\centering
	\begin{subfigure}[b]{0.45\textwidth}
		\centering
		\includegraphics[width=\linewidth]{/home/akai/Downloads/train_loss.png}
		\caption{Pérdida durante el entrenamiento}
		\label{fig:train_loss}
	\end{subfigure}
	\hfill
	\begin{subfigure}[b]{0.45\textwidth}
		\centering
		\includegraphics[width=\linewidth]{/home/akai/Downloads/val_loss.png}
		\caption{Pérdida durante validación}
		\label{fig:val_loss}
	\end{subfigure}

	\begin{subfigure}[b]{0.45\textwidth}
		\centering
		\includegraphics[width=\linewidth]{/home/akai/Downloads/train_plate_acc.png}
		\caption{Precisión durante el entrenamiento}
		\label{fig:train_accuracy}
	\end{subfigure}
	\hfill
	\begin{subfigure}[b]{0.45\textwidth}
		\centering
		\includegraphics[width=\linewidth]{/home/akai/Downloads/val_plate_acc.png}
		\caption{Precisión durante validación}
		\label{fig:val_accuracy}
	\end{subfigure}

	\caption{Curvas de aprendizaje del modelo OCR}
	\label{fig:learning_curves_ocr}
\end{figure}

\begin{figure}[H]
	\begin{subfigure}[b]{0.5\textwidth}
		\centering
		\includegraphics[width=\linewidth]{/home/akai/Downloads/hypr_plate2val_acc.png}
		\caption{Hiperparámetros de imagen contra precisión en validación}
		\label{fig:hypr_plate2val}
	\end{subfigure}
	\begin{subfigure}[b]{0.5\textwidth}
		\centering
		\includegraphics[width=\linewidth]{/home/akai/Downloads/hypr_model2val_acc.png}
		\caption{Hiperparámetros del modelo contra precisión en validación}
		\label{fig:hypr_model2val}
	\end{subfigure}
	\begin{subfigure}[b]{0.5\textwidth}
		\centering
		\includegraphics[width=\linewidth]{/home/akai/Downloads/hypr_train2val_acc.png}
		\caption{Hiperparámetros del entrenamiento contra precisión en validación}
		\label{fig:hypr_train2val}
	\end{subfigure}
	\caption{Optimización de hiperparámetros}
	\label{fig:hypr_opt}
\end{figure}

\subsection{Ejemplo de reconocimiento}

\begin{figure}[H]
	\centering
	\begin{subfigure}[b]{0.45\textwidth}
		\centering
		\includegraphics[width=\linewidth]{/home/akai/Pictures/Plate recognition_screenshot_8.png}
		\caption{Caso de éxito}
	\end{subfigure}
	\hfill
	\begin{subfigure}[b]{0.45\textwidth}
		\centering
		\includegraphics[width=\linewidth]{/home/akai/Pictures/Plate recognition_screenshot_18.png}
		\caption{Caso de éxito con código vertical}
	\end{subfigure}

	\begin{subfigure}[b]{0.45\textwidth}
		\centering
		\includegraphics[width=\linewidth]{/home/akai/Pictures/Plate recognition_screenshot_5.png}
		\caption{Caso de fallo en un carácter}
	\end{subfigure}
	\hfill
	\begin{subfigure}[b]{0.45\textwidth}
		\centering
		\includegraphics[width=\linewidth]{/home/akai/Pictures/Plate recognition_screenshot_7.png}
		\caption{Caso de fallo completo (caso extremo)}
	\end{subfigure}

	\caption{Ejemplos de reconocimiento de placa}
	\label{fig:eje_ocr}
\end{figure}

\section{Integración en prototipo de App}

\begin{figure}[H]
	\centering
	\includegraphics[width=\linewidth]{/home/akai/Pictures/app_screenshot.png}
	\caption{Interfaz de la App}
	\label{fig:app_interface}
\end{figure}

\end{document}
